
\begin{longtable}
       {|>{\setlength\hsize{0.35\hsize}}Y%
        |>{\setlength\hsize{0.65\hsize}}Y|}
\multicolumn{2}{c}{\Large (continued)}\\
\multicolumn{1}{c}{Type} & \multicolumn{1}{c}{Description}\\
\endhead
\multicolumn{2}{c}{\Large Simple Types with their Restrictions}\\
\multicolumn{1}{c}{Type} & \multicolumn{1}{c}{Description}\\
\endfirsthead
\multicolumn{2}{l}{to be cont'd on next page}
\endfoot
\endlastfoot
\hline


\textbf{String} 
    \footnotesize \newline base is xsd:string
      \newline --pattern: {\scriptsize\verb"[^|]*[\w-[|]][^|]*"}  &  String follows the 
         ADES specification that that the pipe character is disallowed in PSV.
         To allow data conversion from XML, it must disallowed in XML as well.
         Also disallow blank elements.  Therefore, all elements must match
         this pattern  \\ 
\hline

\textbf{RightAscension} 
    \footnotesize \newline base is xsd:decimal
      \newline --minInclusive: {\scriptsize\verb"0.0"} 
      \newline --maxExclusive: {\scriptsize\verb"360.0"}  &  RA in \si{\degree} limited to [0.0, 360.0)  \\ 
\hline

\textbf{Declination} 
    \footnotesize \newline base is xsd:decimal
      \newline --minInclusive: {\scriptsize\verb"-90.0"} 
      \newline --maxInclusive: {\scriptsize\verb"90.0"}  &  $\delta$ in \si{\degree} in range [-90.0, 90.0]  \\ 
\hline

\textbf{PositiveDecimal} 
    \footnotesize \newline base is xsd:decimal
      \newline --minExclusive: {\scriptsize\verb"0.0"}  &  PositiveDecimal in range [0.0, +inf)  \\ 
\hline

\textbf{Logical} 
    \footnotesize \newline base is xsd:integer
      \newline --enumeration: {\scriptsize\verb"0"} 
      \newline --enumeration: {\scriptsize\verb"1"}  & 0 for false, 1 for true to match C and FORTRAN \\ 
\hline

\textbf{SelRes} 
    \footnotesize \newline base is xsd:string
      \newline --enumeration: {\scriptsize\verb"A"} 
      \newline --enumeration: {\scriptsize\verb"D"} 
      \newline --enumeration: {\scriptsize\verb"a"} 
      \newline --enumeration: {\scriptsize\verb"d"}  &  SelRes must be ``A," (automatic accept) ``a," (manual accecpt)``D," (automatic delete) or ``d" (manual delete) \\ 
\hline

\textbf{Mode} 
    \footnotesize \newline base is xsd:string
      \newline --enumeration: {\scriptsize\verb"CCD"} 
      \newline --enumeration: {\scriptsize\verb"Photo"}  &  Mode must be ``CCD" or ``Photo"  \\ 
\hline

\textbf{Disc} 
    \footnotesize \newline base is xsd:string
      \newline --enumeration: {\scriptsize\verb"*"}  &  Used to mark the discovery record -- must be `*' if present \\ 
\hline

\textbf{Sys} 
    \footnotesize \newline base is xsd:string
      \newline --enumeration: {\scriptsize\verb"WGS84"} 
      \newline --enumeration: {\scriptsize\verb"ITRF"} 
      \newline --enumeration: {\scriptsize\verb"IAU"} 
      \newline --enumeration: {\scriptsize\verb"ICRF_AU"} 
      \newline --enumeration: {\scriptsize\verb"ICRF_KM"}  &  Coordinate system for station coordinates. This used by the pos[123]  and poscov[123][123] 
         elements to determine the meaning of cooridnates. WGS84, ITRF and IAU are for roving stations,
         ICRF\verb|_|AU and ICRF\verb|_|KM are for fixed stations.   \\ 
\hline

\textbf{FormatType} 
    \footnotesize \newline base is xsd:string
      \newline --enumeration: {\scriptsize\verb"IAU2015"}  &   allowed submission formats with no extra fields  \\ 
\hline

\textbf{MPC80ColFormat} 
    \footnotesize \newline base is xsd:string
      \newline --enumeration: {\scriptsize\verb"MPC1992"} 
      \newline --enumeration: {\scriptsize\verb"MPC1947"}  &  MCP1992 format with extra fields precTime, precRA and precDec \\ 
\hline

\textbf{PermID} 
    \footnotesize \newline base is xsd:string
      \newline --pattern: {\scriptsize\verb"(\d+([A-Z]*(-[A-Z]*)?)?)|([A-Z]+\d*)"}  &  A permID (permanent ID) string may be a positive integer, a positive integer 
       followed by upple-case letters, upper-case letters followed by
       a positive integer, or a positive integer followed by upper-case 
       letters followed by a hyphen followed by more upper-case letters.
       That is, ``134340," ``1P," ``73P-C," ``83P-AC," and ``J13" are all allowed.   \\ 
\hline

\textbf{ProvID} 
    \footnotesize \newline base is xsd:string
      \newline --pattern: {\scriptsize\verb"([PCS]/)?\d{4} ([-A-Z \d]*)"}  &  A provID (provisional ID) is based on a four-digit year number 
      possibly prefixed by P/, C/ or S/ followed by a space and then
      some combination of upper-case lettesr, digits, hyphens and spaces.
      That is, ``2014 AA," ``2001 P-L," ``S/2001 S 31," ``P/1886 S1"
      Perhaps a clearer description would result in a better
      regular exparession  \\ 
\hline

\textbf{Remark} 
    \footnotesize \newline base is String
      \newline --maxLength: {\scriptsize\verb"200"}  &  A remark is a String limited to 200 characters \\ 
\hline

\textbf{FixedStation} 
    \footnotesize \newline base is xsd:string
      \newline --enumeration: {\scriptsize\verb"123"} 
      \newline --enumeration: {\scriptsize\verb"456"} 
      \newline --enumeration: {\scriptsize\verb"F51"}  &  The MPC maintains a list of allowed FixedStations.   at http://somewhere/stationary \\ 
\hline

\textbf{RoverStation} 
    \footnotesize \newline base is xsd:string
      \newline --enumeration: {\scriptsize\verb"427"}  &  The MPC maintains a list of allowed RoverStations.   at http://somewhere/rovers \\ 
\hline

\textbf{RadarStation} 
    \footnotesize \newline base is String
      \newline --maxLength: {\scriptsize\verb"5"}  & MPC maintains a list of radar stations at http://somewhere/radarstations \\ 
\hline

\textbf{AstCat} 
    \footnotesize \newline base is String
      \newline --maxLength: {\scriptsize\verb"5"}  & MPC maintains a list of astrometry catalogs at http://somewhere/astrometrycats \\ 
\hline

\textbf{Band} 
    \footnotesize \newline base is String
      \newline --maxLength: {\scriptsize\verb"5"}  & MPC maintains a list of bands for magnitude observations at http://somewhere/bands \\ 
\hline

\textbf{PhotCat} 
    \footnotesize \newline base is String
      \newline --maxLength: {\scriptsize\verb"5"}  & MPC maintains a list of phototmetry catalogs at http://somewhere/photometrycats \\ 
\hline


\end{longtable}


